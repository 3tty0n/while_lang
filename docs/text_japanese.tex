\documentclass[a4paper,11pt]{jsarticle}
%\usepackage{latexsym}
%\usepackage{epsfig}
\usepackage[dvipdfmx]{graphicx}
\usepackage{epsfig}
%\usepackage{graphicx}
\usepackage{amsmath}
\usepackage{amssymb}
\usepackage{theorem}
\usepackage{multicol}
\usepackage{mdwlist}
\usepackage{url}
\usepackage[colorlinks=true,urlcolor=cyan,linkcolor=blue,dvipdfmx]{hyperref}
\usepackage[dvipdfmx]{hyperref}
\usepackage{pxjahyper}

\usepackage{subcaption}
\usepackage{listings}
\usepackage{float}
\usepackage{here}

\lstset{
  basicstyle=\ttfamily\small,
}

%\pagestyle{empty}

\title{{\bf 情報システム実験I\\コンパイラ}}
\author{伊澤侑祐}
\date{}

\begin{document}
\maketitle

\section{実験目的}
コンパイラとは、ある言語で書かれたソースコードを読み込んで、コンピュータが実行できる機械語
に翻訳するプログラムです。多くの場合、高級言語と呼ばれるC言語やPascalなど人間が理解しやす
い言語で記述されたソースコードをCPU(中央演算処理装置)が実行できる命令コードに変換するプ
ログラムを指します。コンパイラの開発はむづかしいですが、これまでの研究開 発の中から多くの
体系的な開発手法が提案されています。本実験では、簡単なコンパイラを作成して、開発手法の基本
を学びます。

% 図~\ref{phase}にコンパイル処理の流れを示します。\framebox[1cm][l]{ }で囲まれた各フェーズ
% の具体的な処理については、\ref{text}節に示したテキスト「\textbf{コンパイラI}」と
% 「\textbf{コンパイラII}」を参照して下さい。

本実験の目的は以下の通りです。

\begin{enumerate}
\item Pscal 風の高級言語 (While言語) を仮想的なスタック機械の命令コードに翻訳する仕組みを学ぶ。
\item 仮想的なスタック機械の命令コードを経由してPython命令コードへ翻訳する仕組みを学ぶ。同
  時に、実際にPythonインタプリタでWhile言語コンパイラを動作させてみる。
\item スタック機械である Python インタプリタの仕組みの概観を理解する。
\item Visual Studio Code に慣れる。
\end{enumerate}

\section{テキスト}\label{text}
%下記のテキストを読んで,各自で課題をこなす.
%実習書には,実験の目的,言語の仕様,参考書へのポインタ,レポート内容など実験特有の情報を記述する.

以下の本(\textbf{Pascal}以外)をテキストとして、それぞれを1人1冊貸し出します。
必要なら各自の責任の元に、貸し出し帳に記入の上持ち帰っても構いません。
テキストの参照箇所を示す時、テキスト名を\textbf{コンパイラI}のように略して表記しています。
OCaml については、五十嵐淳教授 (京都大学) の「OCaml爆速入門」を「手を動かしながら」読んで
ください。最初の授業でも解説します。

\begin{description}
  \item[OCaml爆速入門]
    「\href{https://www.fos.kuis.kyoto-u.ac.jp/~igarashi/class/pl/03-ocaml.html}{OCaml爆速
      入門}」、五十嵐淳、2019年
  \item[コンパイラI] 「コンパイラ I 言語・技法・ツール」,A.V.エイホ他著,サイエンス社,1990年
  \item[コンパイラII] 「コンパイラ II 言語・技法・ツール」,A.V.エイホ他著,サイエンス社,1990年
  \item[Pascal] 「Pascal」,K.イェンゼン他著,培風館,1993年
\end{description}

\section{ソース言語とスタック機械の仕様}

\subsection{ソース言語の仕様}
%テキストの909ページ付録に構文例がのっているので参考にする.
ソースコードを記述する言語をソース(原始)言語と呼びます。
本実験では、図\ref{spec}の仕様を持つPascal風のソース言語のコンパイラを作成します。
仕様の見方については、\textbf{コンパイラI}の2章2.2節および3章3.3節を参照して下さい。
この書き換え規則により、例えば代入文「\textsf{i:=i+1}」が生成できます(図\ref{生成例})。
また、図\ref{sample}に$9$の階乗を計算するプログラムの例を示します。

\begin{figure}[H]
  \centering
  \textsf{\small
    \begin{tabular}{lcl}
      \hline
      stmt\_list & := & stmt\_list stmt; $|$ stmt; \\
      stmt & := & id := expr $|$ \textbf{while} cond \textbf{do} stmt $|$ \textbf{begin} stmt\_list \textbf{end} \\
      cond & := & expr $>$ expr $|$ expr $<$ expr $|$ expr == expr \\
      expr & := & expr + term $|$ expr $-$ term $|$ term \\
      term & := & term $\ast$ factor $|$ factor \\
      factor & := & id $|$ num $|$ (expr)\\
      id & := & letter (letter $|$ digit)* \\
      num & := & digit digit* \\
      letter & := & a $|$ b $|$ $\ldots$ $|$ z $|$ A $|$ B $|$ $\ldots$ $|$ Z \\
      digit & := & 0 $|$ 1 $|$ 2 $|$ $\ldots$ $|$ 9\\
      \hline
    \end{tabular}
  }
  \caption{ソース言語の仕様}\label{spec}
\end{figure}

\begin{figure}[H]
  \begin{subfigure}[b]{.4\linewidth}
    \centering
    \textsf{\small
      \begin{tabular}{lcl}
        stmt & $\rightarrow$ & id := expr \\
             & $\rightarrow$ & letter := expr \\
             & $\rightarrow$ & i := expr \\
             & $\rightarrow$ & i := expr + term \\
             & $\rightarrow$ & i := term + term \\
             & $\rightarrow$ & i := factor + term \\
             & $\rightarrow$ & i := id + term \\
             & $\rightarrow$ & i := letter + term \\
             & $\rightarrow$ & i := i + term \\
             & $\rightarrow$ & i := i + factor \\
             & $\rightarrow$ & i := i + num \\
             & $\rightarrow$ & i := i + digit \\
             & $\rightarrow$ & i := i + 1
      \end{tabular}
    }
    \caption{文の生成例}\label{生成例}
  \end{subfigure}\hfill
  \begin{subfigure}[b]{.5\linewidth}
    \centering
    \begin{lstlisting}
      a := 1;
      i := 2;
      while i < 10 do
      begin
        a := a * 1;
        i := i + 1;
      end
    \end{lstlisting}
    \caption{プログラム例}\label{sample}
  \end{subfigure}
\end{figure}

\subsection{スタック機械の仕様}
オブジェクトコードはターゲット機械に依存します。
本実験のターゲット機械は仮想的なスタック機械です。詳細は\textbf{コンパイラI}の2章2.8節を参
照して下さい。表~\ref{stackcommand}にスタック機械の命令セットを示します。なお、Pythonバイ
トコードへの変換のため、教科書にある命令セットを一部変更しています。

\begin{table}[H]
\caption{スタック機械の命令セット} \label{stackcommand}
\centering
\normalsize{
  \begin{tabular}{c|p{30zw}}
    \hline
    \texttt{push} $v$ & $v$をスタックに積む。\\
    \texttt{pop} & スタックの最上段の要素を取り去る。\\
    $+$ & 最上段とその下にある要素を取り出して加算し、結果をスタックに積む。\\
    $-$ & 最上段とその下にある要素を取り出して下の値から上の値を引き,結果をスタックに積む。\\
    $\ast$ & 最上段とその下にある要素を取り出して掛け算し、結果をスタックに積む。\\
    $>$ & 最上段とその下にある要素を取り出し、下の値が上の値より大きい場合は$1$(\texttt{true})、そうでない場合は$0$(\texttt{false})をスタックに積む。\\
    $<$ & 最上段とその下にある要素を取り出し、下の値が上の値より小さい場合は$1$(\texttt{true})、そうでない場合は$0$(\texttt{false})をスタックに積む。\\
    $==$ & 最上段とその下にある要素を取り出し、下の値が上の値と等しい場合は$1$(\texttt{true})、そうでない場合は$0$(\texttt{false})をスタックに積む。\\
    \texttt{rvalue} $l$ & データの格納場所$l$の内容をスタックに積む。\\
    \texttt{lpush} $l$ & 最上段の要素を取り出しデータの格納場所$l$に代入する。\\
    \texttt{copy} & 最上段の値を複写してスタックに積む。\\
    \texttt{label} $l$ & 飛先$l$を示す。それ以外の効果はない。\\
    \texttt{goto} $l$ & 次はラベル$l$をもつ命令から実行を続ける。\\
    \texttt{gofalse} $l$ & スタックの最上段から値を取り去り、その値が$0$なら飛越しをする。\\
    \texttt{gotrue} $l$ & スタックの最上段から値を取り去り、その値が$1$なら飛越しをする。\\
    \hline
  \end{tabular}
}
\end{table}

\subsection{具体例}
目的のコンパイラはソース言語で書かれたプログラムをスタック機械の命令コードに変換します。
代入文、begin-end文、while文の具体例と図\ref{sample}のプログラムの翻訳結果を以下に示します。

\begin{enumerate}
\item 代入文 \\
  \fbox{入力} \texttt{a := a * 2;} \\
  \fbox{出力}
  \begin{lstlisting}
rvalue a
push   2
*
lpush  a
  \end{lstlisting}

\item begin-end文

  \fbox{入力} \texttt{begin a := a * i; i := i + 1; end;}

  \fbox{出力}
  \begin{lstlisting}
rvalue a
rvalue i
*
lvalue a
rvalue i
push   1
lpush  i
  \end{lstlisting}

\item while文

  \fbox{入力} \texttt{while i < 10 do i := i + 1;}

  \fbox{出力}

  \begin{lstlisting}
label L.0
rvalue i
push 10
<
gofalse out
lvalue  i
rvalue  i
push    1
+
lpush   i
goto    L.0
label   L.1
  \end{lstlisting}

\end{enumerate}

\section{演習}

\subsection{準備}

本演習では、GitHubリポジトリを通じて資料を提供します。GitHubリポジトリからファイルをダウン
ロードできるか確認してください。

\subsection{1日目: OCaml入門}

1日目では、まずOCamlに入門します。「\href{https://www.fos.kuis.kyoto-u.ac.jp}{OCaml爆速入門}」を読みながらOCamlプログラムの書き方を学
びましょう。余った時間で「コンパイラI」を読み、コンパイラの仕組みについて理解しましょう。

\begin{enumerate}
\item 「OCaml爆速入門」を読みながらOCamlについて勉強する。特に、以下の章を読み練習問題を解
  く。
  \begin{itemize}
  \item
    \href{https://www.fos.kuis.kyoto-u.ac.jp/~igarashi/class/pl/03-ocaml.html#quick-intro-ocaml}{
      OCaml爆速入門 (その1)}
  \item
    \href{https://www.fos.kuis.kyoto-u.ac.jp/~igarashi/class/pl/03-ocaml.html#quick-intro-ocaml2}{
      OCaml爆速入門 (その2) -- データ構造の基本
    } の「ヴァリアント」、「再履ヴァリアント」
  \item
    \href{https://www.fos.kuis.kyoto-u.ac.jp/~igarashi/class/pl/03-ocaml.html#quick-intro-ocaml2}{
      OCaml爆速入門 (その3) -- 変更可能データ構造
    } の「制御構造」
  \end{itemize}
\item \textbf{コンパイラI}(以下,テキスト)の2章「簡単な1パス コンパイラ」の2.1節から2.5
  節を読んで、コンパイラの仕組みを理解する。
\end{enumerate}

さらに時間が余った場合は、2日目の内容を進めてください。


\subsubsection{レポート}

まず、自分が勉強のために記述したOCamlプログラムとその実行結果のうち、面白いと思ったプログ
ラムを3個選んでレポートにまとめてください。また、以下のOCaml爆速入門の練習問題を解きその解
答をレポートにまとめてください。

\begin{enumerate}
\item 与えられた正整数 $n$ に対、$1^2 + 2^2 + \cdots + n^2$ を
  計算する再帰関数 \verb|square: int -> int| を定義せよ。
  (引数 \verb|n| が負である場合の動作はどう定義してもよい。)
\item 与えられた正整数 $n$ に対し、$n$ 番目のフィボナッチ数を計算する再帰関数 \verb|fib : int -> int| を定義せよ。
  (引数 \verb|n| が負である場合の動作はどう定義してもよい。)
\item 一品が菜食主義者でも食べられるかどうかを判定する関数 \verb|isVeggieDish : dish -> bool| を定義せよ.
\item 定食が菜食主義者でも食べられるかどうかを判定する再帰関数 \verb|isVeggieMenu : menu -> bool| を定義せよ.
\end{enumerate}

\subsection{2日目: 仮想スタック機械への簡易コンパイラの作成}

2日目は、While言語から仮想スタック機械へのコンパイラを製作します。
配布するソースコードは大部分が実装されていますが、パーサーが代入文、\verb|begin| --
\verb|end| 文、そして \verb|while| 文を認識しません。\verb|parser.mly| と \verb|lexer.mll|
を改造し、上記の文を認識できるよう改良してください。

認識するための改良が終わったら、仮想スタック機械へのコンパイラを実装してください。大部分が
実装されていますが、パーサと同様に、仮想命令生成器 (\verb|virtual_stack.ml|) が代入文、
\verb|begin| -- \verb|end| 文、そして \verb|while| 文を認識しません。これらを正しく仮想ス
タック機械の命令へ翻訳できるように \verb|virual_stack.ml| を改良してください。

具体的には、以下の手順で取り組んでください。

\begin{enumerate}
\item \verb|syntax.ml| に \verb|Assign|型、 \verb|Seq| 型、そして\verb|While| 型を追加する
\item \verb|parser.mly| で代入、\verb|begin| -- \verb|end|、そして \verb|while| を認識する
  ためのトークンを定義
\item \verb|lexer.mll| で文字列から定義したトークンへ翻訳するためのルールを定義
\item \verb|parser.mly| でトークンから \verb|syntax.ml| で定義した \verb|Assign|,
  \verb|Seq|, \verb|While| へ翻訳するルールを定義
\item \verb|virtual_stack.ml| で \verb|Assign|, \verb|Seq|, \verb|While| を適切な仮想機械
  命令へ翻訳する
\end{enumerate}

\subsubsection{レポート}

自分が改良した内容をレポートにまとめて提出してください。レポートにする際、実行したWhile言
語ソースコードと実行結果を含めることが望ましい。


\subsection{3日目: Pythonバイトコードコンパイラの作成}

3日目は、2日目までに実装した仮想スタック機械コンパイラをPythonバイトコードへコンパイルしま
す。\verb|emit_pyc.ml| と \verb|assemble_pyc.ml| を用い、Pythonバイトコードへ変換してくだ
さい。変換の後、実際にPythonインタプリタで動かしてみましょう。具体的な手順は以下の通りです。

\begin{enumerate}
\item \verb|./while_lang test/assign.while| と実行し \verb|test/assign.pyc| を生成する
\item \verb|./interpret.py test/assign.pyc| と実行し結果を確認する
\end{enumerate}

\subsubsection{レポート}

\begin{itemize}
\item 実際にPythonバイトコードへ変換する際、どういうルールで翻訳を行ったか説明してください。
\item Pythonのバイトコード命令について調べてまとめてください。具体的には、Pythonのバイトコー
  ドが約何命令存在するのか、そして、自分が興味を持った命令について調べて実際に実行した内容
  をまとめてください。
\end{itemize}

\section{Pythonバイトコードを調べる上での資料}

\begin{itemize}
\item 「\href{https://docs.python.org/ja/3/library/dis.html}{Pythonバイトコードの逆アセンブラー}」
\item 「\href{https://github.com/python/cpython/blob/main/Include/opcode_ids.h}{opcode\_ids.h}」、CPython
\end{itemize}

\end{document}
